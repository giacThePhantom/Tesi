\chapter{Processamento dei dati}
\label{cha:processamento}

  \section{Dati disponibili}
  \label{sec:dati}

    \subsection{Sequenze biologiche}
    \label{subsec:fastq}
    Descrizione dei fastq.

    \subsection{Genoma di riferimento}
    \label{subsec:star-gen}
    Descrizione del genoma di riferimento.

    \subsection{Variant call}
    \label{subsec:vcf}
    Descrizione dei vcf.

    \subsection{Struttura dei geni}
    \label{subsec:gtf}
    Descrizione del gtf.

  \section{Troncatura e allinamento}
  \label{sec:trimm_star}
  Descrizione del processo e perch\`e viene fatto.

    \subsection{Troncatura}
    \label{subsec:trimm}
    Trimmomatic, cosa fa come \`e stato usato.

    \subsection{Allineamento}
    \label{subsect:star}
    STAR, cosa fa come \`e stato usato.

    \subsection{Ordinamento}
    \label{subsec:sorting}
    SAMTOOLS SORT cosa fa come \`e stato usato.

    \subsection{Indicizzazione}
    \label{subsec:indexing}
    SAMTOOLS index cosa fa come \`e stato usato.

  \section{Deduplicazione, riallinamento e recalibrazione}
  \label{sec:recalibration}
  Descrizione del processo e perch\`e viene fatto

    \subsection{Deduplicazione}
    \label{subsec:dedup}
    Come sopra.

    \subsection{Riallineamento e recalibrazione}
    \label{subsec:recalibration}
    Come sopra.

  \section{Ottenere le varianti alleliche}
  Intersezione tra VCF e GTF.

  \section{Ottenere i dati delle frazioni alleliche}
  \label{sec:aseq}
  ASEQ cosa fa come viene usato.


    \subsection{Filtrare le frazioni alleliche}
    \label{subsec:filter}
    Condizioni di filtraggio per i risultati di ASEQ.

  \section{Ottenere gli SNP nel 3'-UTR}
  \label{sec:threeprime}
  Filtraggio del gtf e intersezione con i VCF
