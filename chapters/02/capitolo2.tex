\chapter{Linea cellulare e dati di partenza}
\label{cha:cell_lines}

\section{HCT116}
\label{sec:hct116}
La linea cellulare soggetto di analisi \`e HCT116.
\`E una linea cellulare di un carcinoma colon-rettale umano presente nel pannello originale delle $60$ linee caratterizzate a fondo dall'iniziativa del national cancer institute statunitense (NCI-$60$).
Come descritto in \cite{hct116} la linea \`e perfettamente diploide e mostra un'abilit\`a tumorigenica intermedia: se vengono iniettate in una popolazione di topi nudi atimici $5\cdot 10^6$ cellule il $50\%$ di questi sviluppano un tumore dopo un periodo di latenza di $16$ giorni.
Oltre a queste caratteristiche la linea cellulare \`e wild-type per p53, una proteina soppressore dei tumori in grado di regolare la sintesi proteica in modo da inibire la crescita cellulare.

  \subsection{Controllo traduzionale da parte di p53}
  \label{subsec:p53}
  La proteina p53 soppressore dei tumori \`e il fattore di trascrizione mammifero meglio caratterizzato che media diversi processi anti-proliferativi.
  Ha un effetto importante sull'iniziazione della traduzione e una sua caratterizzazione \`e fondamentale per comprendere il ruolo che sue mutazioni o deregolazioni hanno nella biologia del cancro.
  p53, oltre a regolare la trascrizione, controlla la biogenesi dei ribosomi e dei fattori di iniziazione eucarioti.
  Si analizza pertanto il suo ruolo come regolatore dei complessi ternario e eIF4E e nella biogenesi dei ribosomi.

    \subsubsection{p53 limita la biogenesi dei ribosomi}
    I ribosomi sono responsabili del trasferimento dell'informazione contenuta negli mRNA in proteine.
    La loro biogenesi ha luogo nel nucleolo, in cui il DNA ribosomiale \`e organizzato.
    La subunit\`a $60S$ \`e una molecola complessa composta di $3$ RNA ribosomiali e $47$ prtoeine.
    Questo complesso \`e responsabil e per la formazion edel legame peptidico e del controllo della qualit\`a del peptide nascente.
    La subunit\`a $40S$ invece \`e responsabile per lo svolgimento e la scansione del mRNA ed \`e composta da $1$ rRNA e $33$ prtoeine.
    Disturbi nel processo di biosintesi delle componenti del ribosoma ha un ruolo centrale nella tumorigenesi.\\
    p53 limita questo processo: regola infatti la RNA polimerasi $I$, che sintetizza gli RNA ribosomiali, inibendola.
    Questo processo coinvolge l'interferenza di p53 con un insieme di proteine richieste per l'assemblaggio e l'iniziazione del macchinario trascrizionale sul promotore del gene del rRNA.
    p53 Si lega alla proteina legante la TATA-box e ai suoi fattori associati impedendo la loro interazione con fattori di legame a monte e reprimendo la trascrizione del RNA polimerasi $I$.\\
    Inoltre p53 inibisce l'attivit\`a della RNA polimerasi $III$, diminuendo la produzione di tRNA, del rRNA $5S$ e di altri piccoli RNA coinvolti nel trasporto e nel processamento del RNA impedendo l'attacco della polimerasi al DNA.

    \subsubsection{p53 regola la trascrizione dei geni RP}
    Le proteine ribosomiali o \emph{RP} di nuova sintesi sono importate nel nucleolo dal citosol.
    In risposta a stress nucleolare diverse RP traslocano nel nucleoplasma legandosi e inibendo l'attivit\`a di MDM2 e causando un arresto del ciclo cellulare e apoptosi mediati da p53.
    Inoltre in risposta a danni al DNA la proteina ribosomiale RPL26 si lega alle terminazioni del mRNA di p53 aumentando la sua traduzione e portando all'arresto del ciclo cellulare.
    Infine durante una condizione di stress genotossico p53 induce l'espressione di una proteina ribosomiale che aumenta l'espressione di p21, che media a sua volta l'arresto del ciclo cellulare.\\
    Oltre a regolare la trascrizione di rRNA p53 controlla il processamento dei pre-rRNA inibendo i livelli di espressione di FBL (fibrillarin), una proteina nucleolare vitale per la metilazione e il processamento deti pre-rRNA.
    L'inibizione causa un'alta infedelt\`a della traduzione e aumenta l'iniziazione di traduzione dipendente da \emph{IRES}.

    \subsubsection{p53 regola l'assemblaggio del complesso ternario e di eIF4F}
    p53 \`e in grado di attenuare la sintesi globale di proteine grazie all'inibizione della proteina ribosomiale S6 chinasi, una chinasi a monte di 4E-BP1.
    Inoltre un gene obiettivo di p53, TRIM22 inibisce il legame di eIF4E a eIF4G.
    Inoltre p53 causa defosforilazione e rottura del fattore di iniziazione eIF4GI e di 4E-BP1.
    Si \`e dimostrato come gli effetti combinati dell'assenza delle 4E-BP e di p53 aumentano sinergisticamente la proliferazione cellulare e la tumorigenesi.
    p53 pertanto inibisce la sintesi proteica attraverso l'inibizione di eIF4E e l'assemblaggio del complesso eIF4F aumendando la de-fosforilazione di 4E-BP1.
    Non interagisce con il complesso ternario ma unicamente con diverse componenti del complesso eIF4F.\\
    p53 inibisce anche la segnalazione con mTOR e l'attivit\`a della proteina chinasi CK2, in grado di regolare a sua bolta eIF2$\alpha$ e quindi il complesso ternario.
    CK2 \`e in grado di fosforilare p53 a un residuo altamente conservato, causandone la traslocazione nel mitocondrio e apoptosi indipendente dalla trascrizione dopo l'esposizione della cellula ad agenti genotossici.
    Infine la subunit\`a regolatoria $\beta$ di CK2 \`e in grado di interagire con p53 riducendone la funzione transattivatrice e l'affinit\`a con il DNA.

\section{Culture cellulari}
I dati di RNA-seq sono stati ottenuti per colture di HCT116 in diverse condizioni.
Oltre a un ambiente di controllo in presenza di \emph{DMSO} \`e stata utilizzata anche \emph{Nutlin}, una piccola molecola in grado di attivare p53 e pertanto arresto del ciclo cellulare e apoptosi.
Anche lo stato genetico \`e stato modificato dalla condizione di controllo denominata \emph{scr} \`e stato svolto il knockout di due geni: \emph{PCBP2} e \emph{DHX30}, coinvolti nel processo traduzionale attivato da p53 che porta all'apoptosi.
Si ottengono infine $6$ condizioni diverse su cui viene svolto il profilamento polisomico e da cui si ottengono i dati di RNA-seq:
\begin{table}[H]
  \begin{tabular}{|c|c|c|}
    \hline
    Denominazione & Ambiente & Stato genetico\\
    \hline
    scr\_DMSO & DMSO & Normale\footnote{corretto?}\\
    \hline
    scr\_NUTLIN & NUTLIN & Normale\\
    \hline
    shDHX30\_DMSO & DMSO & knockout di DHX30\\
    \hline
    shDHX30\_NUTLIN & NUTLIN & knockout di DHX30\\
    \hline
    shPCBP2\_DMSO & DMSO & knockout di PCBP2\\
    \hline
    shPCBP2\_NUTLIN & NUTLIN & knockout di PCBP2\\
    \hline
  \end{tabular}
  \centering
  \caption{Condizioni di coltura}
\end{table}

\section{Apoptosi indotta da Nutlin causata da un processo traduzionale regolato da PCBP2 e DHX30}
Come descritto in \S\ref{subsec:p53} p53 \`e una proteina strettamente controllata e altamente pleiotropica tipicamente disattivata nelle cellule tumorali umane.
Tra le varie funzioni svolte da essa quella considerata pi\`u rilevante nel contesto del cancro \`e il controllo della morte cellulare programmata.
In quanto una funzione di p53 non controllata produrrebbe una morte cellulare massiva esiste un MDM2, una proteina agente come una ubiquitina ligasi E3 che ne inibisce l'attivit\`a.
Questa proteina inibitrice \`e tipicamente sovra-espressa nella frazione di tumori che mantengono p53 wild-type, come succede per HCT116.
Un trattamento possibile per questi tipi di tumori va ad attaccare questo meccanismo, in particolare Nutlin inibisce l'interazione tra p53 e MDM2, attivando la prima.
I risultati di tale trattamento variano in una combinazione di arresto del ciclo cellulare, senescenza e apoptosi, in proporzioni di difficile previsione.
Lo studio \cite{nutlin} tenta di determinare le cause dietro le diversit\`a fenotipiche a seguito del trattamento con Nutlin.
Utilizza come modello due linee cellulari: HCT116 e SJSA1, un osteosarcoma maligno.
Se la prima a seguito del trattamento con Nutlin subisce un arresto del ciclo cellulare, la seconda subisce un'apoptosi massiva.

  \subsection{Gli esiti diversi del trattamento nelle due linee sono riflessi da cambi nel traduttoma}
  Se i cambi trascrizionali causati dal trattamento con Nutlin contengono invariantemente geni coinvolti in multipli pathway, si analizza il traduttoma in modo da determinare differenze in esso tra le due linee cellulari.
  Dopo il trattamento per ognuna delle linee cellulari si sono analizzati i dati di RNA-seq frazionati con profilamento polisomico e sono stati identificati diversi geni espressi in maniera diversa rispetto alla condizione di controllo.
  Questi sono indicati come \emph{DEG} e sono stati divisi in $3$ classi:
  \begin{itemize}
    \item Accoppati: DEG in cui cambi nella frazione polisomiale sono accoppiati con cambi nella frazione totale.
    \item Regolati traduzionalmente: DEG che esibiscono cambi unicamente nella frazione polisomiale.
    \item Invariati nella traduzione: DEG che esibiscono cambi unicamente nella frazione sub-polisomiale.
  \end{itemize}
  Tra questi sono stati presi in analisi i DEG regolati traduzionalmente e si \`e notato come i geni associati con la segnalazione apoptotica sono traduzionalmente potenziati unicamente nelle cellule SJSA1, spiegando la loro abilit\`a di avviare l'apoptosi a seguito del trattamento.
  Questo mostra come anche se l'attivazione di p53 da parte di Nutlin attivi programmi trascrizionali nelle due linee cellulari, questi causano programmi traduzionali diversi.

  \subsection{Il potenziamento traduzionale dipendente dalla linea cellulare \`e associato con la presenza di un elemento nella 3'-UTR}
  Data la specificit\`a degli obiettivi regolati traduzionalmente in ogni linea cellulare \`e stato cercato un motivo de novo.
  Questo ha permesso l'identificazione di un elemento cis-regolatorio specificatamente arricchito nella 3'-UTR di geni potenziati traduzionalmente solo nelle cellule SJSA1.
  Il motivo, definito come motivo-CGPD corrisponde al consenso $5'$-$CCCC(A/C)(T/G)GGCCCT$-$3'$.
  Circa il $65\%$ dei geni potenziati traduzionalmente nelle cellule SJSA1 dopo il trattamento presentano almeno una copia del motivo-CGPD.
  In HCT116 invece l'espressione dei geni contenenti il motivo o geni-CGPD non \`e attivata da p53.\\
  Ulteriori analisi hanno mostrato come l'effetto maggiore si ottiene quando il motivo-CGPD \`e presente in due copie nel mRNA.
  Si \`e notato come il motivo-CGPD \`e sufficiente per potenziare la traduzione di un mRNA reporter in cellule SJSA1 e conferisce una tendenza alla repressione traduzionale nelle cellule HCT116.

  \subsection{Le proteine PCBP2 e DHX30 legano il motivo CGPD in maniera dipendente dalla linea cellulare}
  Per scoprire il meccanismo sottostante il risultato diversificato del trattamento con Nutlin nelle due linee cellulari si analizzano proteine leganti il RNA capaci di riconoscere il motivo-CGPD in maniera dipentente dal ciclo cellulare.
  Ci si aspetta che queste proteine abbiano modelli di espressioni opposti tra le due linee cellulari.
  Questo ha portato all'identificazione di DHX30 tra gli estratti di HCT116 che ha mostrato un cambio di espressione tra il mRNA polisomiale e tra i livelli di proteina.
  Inoltre si \`e analizzata PCBP2, facente parte della famiglia di proteine PCBP, capaci di creare forti legami con lunghezze poli-C come quella presente nel motivo.
  Quest'ultima presenta maggiori livelli di espressione a livello proteico in HCT116 rispetto a SJSA1, suggerendo un suo ruoo nella regolazione del motivo-CGPD in HCT116.
  DHX30 e PCBP2 sono state sottoposte ad ulteriori analisi e si \`e notato come gli mRNA che presentano un aumento del potenziale di traduzione in SJSA1 sono obiettivi di queste due proteine.
  Questo aumento non \`e cos\`i significativo invece in HCT116.
  Si \`e anche notata una sovrapposizione tra i siti di legame di DHX30, PCBP2 e il motivo-CGPD.
  Questo suggerisce PCBP2 e DHX30 come probabili candidati capaci di legare il motivo-CGPD in vivo.
  Inoltre le due proteine sono espresse a diversi livelli nelle due linee cellulari, suggerendo che potrebbero contribuire alla repressione traduzionale di mRNA contenenti il motivo nelle cellule HCT116.
  Ulteriori anali hanno dimostrato ocme il legame di DHX30 al motivo-CGPD avviene principalmente nelle cellule HCT116, mentre PCBP2 in entrambe.
  Viene suggerito come PCBP2 agisce come fattore di vincolo al motivo-CGPD in quanto il suo legame \`e dipendente dalla sequenza ma indipendente dalla linea cellulare.
  L'espressione dipendente dalla linea cellulare di DHX30 potrebbe agire come repressore dell'efficienza di traduzione di mRNA contenenti il motivo in HCT116.
  Non si pu\`o per\`o escludere che DHX30 sia in grado di legare il motivo indipentemente da PCBP2.
  Si pu\`o osservare come il legame di queste due proteine \`e potenzialmente coinvolto nel controllo traduzionale di mRNA contenenti il motivo-CGPD.





\section{Dati di partenza}
\label{sec:datiinput}
I dati di partenza necessari per l'analisi di sbilanciamento allelico comprendono:
\begin{itemize}
  \item I dati di RNA-seq per ogni condizione della linea cellulare divisi tra frazione allelica e totale.
  \item I dati di whole exome sequencing contenenti informazioni riguardo agli SNP presenti nella linea cellulare, divisi tra file GTF e VCF.
\end{itemize}

  \subsection{Dati di RNA-seq}
  Dove sono stati ottenuti i dati di RNA-seq.

  \subsection{Dati WES}
  Dove sono stati ottenuti i dati WES.
