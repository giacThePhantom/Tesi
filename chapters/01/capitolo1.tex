\chapter{Introduzione}
\label{cha:intro}
%Replicatre \cite{transsnp} su altra linea cellulare.
%Sfrutta la grande mole di dati disponibili grazie al NGS.
%Identificazione di nuova classe di SNP detti transnp.
%Cosa sono gli SNP, come si identificano questi particolari.
%Perch\`e sono importanti con esempi.




Il progetto di ricerca presentato in questo elaborato tenta di replicare il processo presentato in \cite{transsnp} su un'altra linea cellulare: \emph{HCT116}\footnote{Riferimento a linea cellulare}.
Questo viene fatto in modo da eliminare la possibilit\`a che il fenomeno osservato sia specifico a \emph{MCF7}.
Sempre per determinare se l'evento \`e estensibile al di fuori dell'essere umano il progetto \`e stato replicato in parallelo su una linea cellulare murina \emph{B16-F10}\footnote{Ce lo metto? Devo citare in qualche modo?}.
In particolare si tenta di determinare una nuova classe di mutazioni dette \emph{transSNP} che potrebbero essere in grado di avere effetto sui livelli di espressione di proteine e che potrebbero essere una fonte potenziale di variazione inter-individuale nel richio di cancro\footnote{Magari qua ci va un termine tecnico?}.

\section{Polimorfismi a singolo nucleotide}
I polimorfismi a singolo nucleotide, da qui in avanti denominati \emph{SNP} sono delle mutazioni nel genoma causate dal cambio di un singolo nucleotide nella molecola di DNA.
Sono una delle classi pi\`u grandi di variabilit\`a genetica che possono sottostare o essere responsabili di variazioni inter-individuali in complessi fenotipi di malattie.
Grazie allo sviluppo tecnologico nelle tecnologie di sequenziamento e alla nascita del next-generation sequencing sono state rese disponibili informazioni a livello della singola base sul genoma umano e sul trascrittoma permettendo l'esplorazione di questioni biologiche prima insondabili.





Questo capitolo \`e volto a descrivere i processi biologici considerati durante il progetto.
Cito principalmente dal draft paper sui transSNPS

\section{TransSNPs}
\label{sec:transsnps}
Definizione di SNP e loro impatto.
Descrizione degli SNP considerati in questo esperimento.

\section{Sbilanciamento allelico}
\label{sec:allelicimbalance}
Definizione di sbilanciamento allelico e perch\`e viene considerato.

	\subsection{Profilamento polisomico}
	\label{subsec:polysomalprofiling}
	Come si identifica lo sbilanciamento allelico.
