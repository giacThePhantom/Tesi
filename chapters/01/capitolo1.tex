\chapter{Introduzione}
\label{cha:intro}
%Replicatre \cite{transsnp} su altra linea cellulare.
%Sfrutta la grande mole di dati disponibili grazie al NGS.
%Identificazione di nuova classe di SNP detti transnp.
%Cosa sono gli SNP, come si identificano questi particolari.
%Perch\`e sono importanti con esempi.




Il progetto di ricerca presentato in questo elaborato tenta di replicare il processo presentato in \cite{transsnp} su un'altra linea cellulare: \emph{HCT116}\footnote{Riferimento a linea cellulare}.
Questo viene fatto in modo da eliminare la possibilit\`a che il fenomeno osservato sia specifico a \emph{MCF7}.
Sempre per determinare se l'evento \`e estensibile al di fuori dell'essere umano il progetto \`e stato replicato in parallelo sulla linea cellulare murina \emph{B16-F10}\footnote{Ce lo metto? Devo citare in qualche modo?}.
In particolare si tenta di determinare una nuova classe di mutazioni dette \emph{transSNP} che potrebbero essere in grado di avere effetto sui livelli di espressione di proteine controllandone la traduzione e perci\`o essere una fonte potenziale di variazione inter-individuale nel richio di cancro\footnote{Magari qua ci va un termine tecnico?}.

\section{Controllo traduzionale nel cancro}
La traduzione di mRNA in proteine \`e un evento chiave nella regolazione dell'espressione genica.
Questo \`e specialmente vero nel contesto del cancro in quanto molti oncogeni ed eventi di trasformazione sono regolati a questo livello.
In \cite{transconcancer} vengono esplorati i diversi modi in cui le cellule del cancro deregolano e riprogrammano la traduzione e il loro impatto oncogenico.
Risorse considerevoli sono dedicate alla traduzione di mRNA in cellule normali.
Fino al $20\%$ dell'energia cellulare viene impiegata nella sintesi delle proteine e anche la maggior parte della trascrizione \`e volta alla produzione di RNA ribosomiale e di mRNA codificante proteine ribosomiali.
La proliferazione di cancri maligni richiede una sintesi continua di proteine e un aumento del contenuto di ribosomi.
Molte cellule tumorali subiscono stress fisiologico come ipossia o mancanza di nutrienti.
In queste condizioni l'efficienza della traduzione si riduce, ma i meccanismi id regolazione vengono disaccoppiati dal processo nelle cellule tumorali come conseguenza del processo di trasformazione, aumentando lo stress della cellula.
Si nota come le cellule tumorali possono sfruttare il meccanismo di traduzione per sostenere la loro proliferazione, sopravvivenza e metastasi.
Questo avviene cambiando l'attivit\`a e l'espressione di fattori di traduzione che conferiscono alla cellula una capacit\`a di traduzione di mRNA specifica per il cancro.
Dati i tassi di allungamento del ribosoma sul RNA costanti e i fattori di iniziazione di tradizione limitati, la traduzione di mRNA specifici per cui il reclutamento del ribosoma \`e inefficiente viene disproporzionalmente affetta da cambi nell'attivit\`a dei fattori di iniziazione.
Da questo si nota come le cellule tumorali possiedono una variet\`a di complesse alterazioni molecolari che aumentano la traduzione selettiva di mRNA mal tradotti\footnote{Il paper intendeva trascritti?}.
In particolare aumenta l'espressione o la disponibilit\`a di fattori di iniziazione della traduzione specifici e l'attivit\`a dei pathways di segnalazione che li regolano.
L'espressione aberrante dei fattori di iniziazione di traduzione \`e il primo meccanismo scoperto attraverso cui le cellule di cancro deregolano la traduzione.
\`E stato mostrato originariamento dall'abilit\`a di eIF4E sovra-espresso di trasformare le cellule NIH 3T3.
Ulteriori fattori di iniziazione sono stati scoperti in tumori umani.

	\subsection{Formazione del complesso eIF4F}
	I ribosomi sono reclutati alla terminazione $5'$ del mRNA attraverso il complesso eIF4F, composto di tre subunit\`a, tra cui eIF4A che fornisce l'attivit\`a elicasica necessaria per svolgere le strutture secondarie presenti nella $5'$-UTR.
	Questo processo \`e aiutato dalle altre due subunit\`a eIF4G e eIF4E.
	Considerando che tipicamente tale regione degli mRNA oncogenici \`e lunga e stabile questi sono molto sensibili all'attivit\'a di eIF4A e alla formazione di eIF4F.
	Tutte e tre le subunit\`a possono essere deregolate nelle cellule cancerogene: i loci genomici sono amplificati in molti tumori umani e sono tutti obiettivi dell'oncoproteina MYC.
	La sovra-espressione di eIF4E e eIF4G, che agiscono come classici oncogeni, risulta nella trasformazione delle cellule in vitro e in vivo.\\
	La regolazione dell'iniziazione della traduzione nel cancro pi\`o essere anche regolata alla forsforilazione del complesso eIF4F.
	La forsforilazione di eIF4E promuove lo sviluppo di tumori e la loro disseminazione ed \`e elevata in tumori umani di polmoni, prostata e seno.
	Un sito di fosforilazione di eIF4G quando legato promuove la fosforilazione di eIF4E, coinvolta nella riprogrammazione traduzionale che porta alla resistenza al tamoxifene nel cancro al seno.
	Questo meccanismo \`e poco compreso, ma si pensa coinvolga il riciclo dei fattori di iniziazione.\\
	Un ulteriore meccanismo di regolazione coinvolge il sequestro dei fattori di iniziazione per impedire la formazione del complesso eIF4F.
	Questo avviene da parte di PDCD4 (tumor suppressor programmed cell death 4), la cui perdita \`e associata con un invasione della cellula tumorale e un abbassamento della probabilit\`a di sopravvivenza per alcuni tumori.
	Un meccanismo simile coinvolge 4E-BP, che competono con eIF4G per il legame con eIF4E, inibendo la traduzione dipendente dal cap.
	La loro espressione pu\`o essere persa o la loro funzione inibita attraverso fosforilazione.
	4E-BP possono contrastare la metastasi, ma facendolo promuovono lo sviluppo di grandi tumori locali.

	\subsection{Formazione del complesso ternario}

	\subsection{Connessione di eIF4F e eIF3 e formazione del complesso di pre-iniziazione}

	\subsection{Allungamento e terminazione della traduzione}


	\subsection{Cambi nelle regioni UTR nelle cellule di cancro}
	Come evidenziato in \cite{tranconcancer} sequenze e motivi strutturali presenti negli mRNA determinano una loro efficienza traduzionale e la loro abilit\`a di essere regolati da fattori agenti in trans come microRNA, proteine leganti RNA e fattori di iniziazione.
	Questi elementi si trovano nelle regioni $5'$ e $3'$ UTR e tendono ad essere sovra-rappresentati in mRNA oncogenici garantendo una loro precisa regolazione.
	\`E stato mostrato come mutazioni come gli SNP in questi motivi non codificanti possono modulare in maniera significativa l'espressione di proto-oncogeni.\\
	L'aumento di strutture secondarie nel $5'$ UTR ha effetto sul tasso di iniziazione della traduzione di mRNA cap-dipendente.
	In particolare si nota come mRNA oncogentici possiedono strutture stabili nella $5'$ UTR e hanno una maggiore dipendenza da \emph{eIF4F}.
	Altri elementi nelle due UTR possono regolare l'efficenza della traduzione: una maggiore dipendenza da \emph{eIF4E} ma non da \emph{eIF4A} \`e stata mostrata per l'elemento iniziatore di traduzione del $5'$ UTR corto di alcuni mRNA.\\
	In contrasto mRNA contenenti siti di ingresso di ribosomi interni o \emph{IRES} sono altamente dipendenti da \emph{eIF4G} e \emph{eIF4A}.
	Inoltre gli mRNA possono contenere codoni di iniziaizone alternativi e open reading frames \emph{ORF} inibitori a valle del codone di inizio canonico che possono severamente contrastare la normale identificazione del normale sito di inizio di traduzione.
	Questi elementi di sequenza sono arricchiti di trascritti oncogenici e in condizioni di stress oncogenico alcuni di questi mRNA mostrano una traduzione aumentata.\\
	Oltre a questo motivi di sequenza o strutturali in alcune delle $5'$ UTR mediano il reclutamento di proteine leganti RNA che modulano la sua traduzione.
	Un esempio ben caratterizzato di questo \`e l'elemento ditraduzione attivata dal transforming growth factor $\beta$ \emph{TGF-$\beta$} che regola la traduzione di certi mRNA coinvolti nella transizione da cellula epiteliale a mesenchimale promuovendo la migrazione cellulare.\\
	Infine un altro elemento da tenere in conto sono i siti di legame per i microRNA, motivi particolarmente comuni con un effetto sulla traduzione e sulla stabilit\`a del mRNA.
	La maggior parte di questi elementi riduce l'efficienza di traduzione del mRNA e si trovano in varie combinazioni in mRNA oncogenici attenuandone la traduzione in modo da impedire la trasformazione della cellula causata da una loro sovra-espressione.
	Nonostante tutto questo le cellule di cancro riescono a trovare meccanismi in grado di superare questi controlli.\\
	Si nota pertanto come nelle regioni non codificanti degli mRNA si trovano motivi e sequenze di fondamentale importanza per il benessere della cellula e mutazioni come gli SNP possono andare a distruggerli o ad intaccarne l'efficacia, rendendola pertanto pi\`u prona a trasformarsi in una cellula di cancro.

	\subsection{Segnalazione oncogenica}

	\subsection{Ribosoma tumorale}

	\subsection{Vantaggi oncogenici selettivi di una traduzione deregolata}

	\subsubsection{Proliferazione e apoptosi}

	\subsubsection{Angiogenesi}

	\subsubsection{Risposta allo stress}

\section{Polimorfismi a singolo nucleotide}
I polimorfismi a singolo nucleotide, da qui in avanti denominati \emph{SNP} sono delle mutazioni nel genoma causate dal cambio di un singolo nucleotide nella molecola di DNA presenti in almeno $1\%$ della popolazione.
Sono una delle classi pi\`u grandi di variabilit\`a genetica che possono sottostare o essere responsabili di variazioni inter-individuali in fenotipi complessi di malattie.
Grazie allo sviluppo tecnologico nelle tecnologie di sequenziamento e alla nascita del next-generation sequencing sono state rese disponibili informazioni a livello della singola base sul genoma umano e sul trascrittoma permettendo l'esplorazione di questioni biologiche prima insondabili.
Infatti diversi strumenti sono stati implementati per studiare i dati di espressione genica basati su RNA-sequencing in modo da identificare istanze di espressione genica allelo-specifica.

\subsection{Espressione genica allelo-specifica}
Si intende per espressione genica allelo-specifica o \emph{ASE} una condizione per cui alleli diversi di un gene, per lo scopo di questo progetto uno contenente uno SNP e uno no, mostrano un'attivit\`a trascrizionale considerevolmente diversa.
Un evento di ASE si osserva pertanto nelle cellule umane dove la trascrizione si origina principalmente da un allele.
Questi fenomeni sono dovuti principalmente a geni imprinted, condizioni fisiologiche come l'inattivazione del cromosoma \emph{X} e contribuiscono alla variabilit\`a fenotipica umana.
Ulteriori meccanismi comprendono degradazione dei trascritti da parte di miRNA, distruzione monoallelica di regioni regolatorie, pattern di splicing alternativi o fenomeni epigenetici.

\section{TransSNP}
Una frazione di SNP identificati nella popolazione umana sono locati nelle regioni codificanti o negli \emph{UTR}.
In questo caso studi guidati da meccanismo e da associazione hanno tentato di studiare SNP funzionali che possono modificare aspetti di regolazione genica post-trascrizionale.
Non sono per\`o stati ancora esplorati SNP associati con alterazioni nel potenziale di traduzione del mRNA, estendendo il concetto di ASE dall'aspetto trascrizionale all'aspetto traduzionale.
Lo scopo del progetto \`e pertanto identificare questi ultimi, SNP in grado di cambiare l'efficienza della traduzione del mRNA che li contiene, e una loro eventuale correlazione con il cancro, denominati transSNP.
Si nota infatti come la regolazione traduzionale governa la produzione di proteine in risposta a un gran numero di situazioni fisiologiche e patologiche: circa met\`a della variazione della concentrazione di una proteina \`e dovuta a questo tipo di controllo.
Per farlo si utilizza un'analisi comparativa di sbilanciamento allelico tra frazioni di mRNA totali e polisomiali estratti dallo stesso campione cellulare in modo da superare il rumore causato dalla chiamata degli SNP e dal coverage derivato dai dati di RNA-seq.
Questo tipo di analisi viene svolto unicamente su SNP in eterozigosi nella cellula: in questo modo si ottiene la percentuale di allele presente in una frazione rispetto all'altro\footnote{Non riesco a spiegare bene il concetto}.
In questo modo si crea un catalogo di SNP codificanti e negli UTR associati e cause potenziali con alterazioni nel potenziale di traduzione degli mNRA.


\subsection{Profilamento polisomico}
Il profilamento polisomico \`e il metodo con cui le frazioni di mRNA totali e polisomali vengono separate in un campione e il suo protocollo viene descritto in \cite{polprofiling}.
Permette di determinare il sottoinsieme di mRNA attivamente coinvolti nella traduzione o traduttoma, ritornando una visione funzionale del genoma in un dato momento in una data cellula.
Questo metodo offre diversi vantaggi rispetto ad altri, per esempio, a differenza del profilamento a ribosomi questa tecnica d\`a accesso all'intera lunghezza degli mRNA, comprese le UTR, le regioni che questo progetto vuole analizzare.
La separazione delle due frazioni si basa su una centrifugazione con gradiente: i ribosomi hanno un coefficiente di sedimentazione molto maggiore rispetto alle molecole di mRNA e pertanto si troveranno ad altezze diverse della colonna.
Pertanto le cellule vengono lisate e i lisati citoplasmatici vengono caricati su un gradiente di saccarosio lineare $10-50\%$, ultra-centrifugate e frazionate attraverso un collettore automatico di frazioni che tiene conto dell'assorbanza a $254nm$.
Tutte le parti pi\`u leggere contenenti frazioni subpolisomali presenti dalla cima fino alla frazione corrispondente al monosoma $80S$ sono assunte non attivamente coinvolte nel processo di traduzione e raccolte in una provetta.
Le frazioni pi\`u pesanti sono quelle attivamente tradotte e sono raccolte in una seconda provetta\footnote{Provetta \`e il termine giusto? Nel paper si parla di ``tube''}.
Successivamente le molecole di RNA sono purificate e sospese in acqua sterile.
In questo modo la seconda provetta contiene la frazione polisomale di interesse per il progetto.\\
La frazione totale viene ottenuta attraverso un'estrazione con TRIzol (ThermoFisher) di una popolazione cellulare separata preparata in parallelo\footnote{Sto prendendo dal paper, magari per la mia linea cellulare queste cose sono state fatte in modo diverso}.\\
In questo modo sono state ottenute tutte le due frazioni di mRNA, prima la polisomale e poi la totale che verranno sequenziate in modo poi da poter studiare lo sbilanciamento allelico al loro interno.
Valori diversi di ASE tra le due frazioni indicano un cambio nel potenziale di traduzione causato dallo SNP considerato.

\subsection{Sequenziamento}
Le frazioni ottenute attraverso il profilamento polisomico vengono poi sequenziate attraverso \emph{HiSeq 2500} di Illumina.
Le molecole di RNA vengono frammentate e convertite in cDNA a cui viene aggiunta una sequenza adattatrice.
Successivamente avviene un'amplificazione con \emph{PCR} i cui risultati sono caricati in una \emph{flow cell} dove i frammenti sono catturati da oligonucleotidi legati alla superficie complementari agli adattatori.
Si formano in questo modo dei cluster di frammenti che presentano la stessa sequenza adattatrice.
Successivamente i reagenti di sequenziamento, che includono nucleotidi etichettati con un nucleotide fluorescente sono aggiunti in modo da incorporare la prima base.
La flow cell \`e viene letta dalla macchina che registra la lunghezza d'onda emessa dai cluster.
La particolare lunghezza d'onda permette di identificare il nucleotide.
Il ciclo viene poi ripetuto $n$ volte in modo da creare una read lunga $n$ basi.
In questo modo si ottengono per ogni frazione i file \emph{fastq} poi utilizzati per l'analisi dell'espressione allelo-specifica.
